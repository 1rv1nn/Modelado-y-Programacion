\documentclass{bachw}

% Author and title
\author{Cerebros de Pollo}
\title{\textbf{Puppy Pets}\\Proyecto 2}

\begin{document}

\textbf{Categoria:Negocio} \\

Implementar un sistema de prueba de un Hospital Veterinario que lleva por nombre \textbf{Puppy Pets}\\

El sitio debe guardar información valiosa de las mascotas(ID,nombre,nombre del propietario,especie,género,raza si es gato o perro),
de los propietarios o clientes(nombre,telefono,\\dirección,cuenta bancaria asociada) y de los médicos veterinarios(ID,nombre,género).

El rol de médico o cliente servirá para cambiar la interfaz del sitio,el sitio funcionará para médicos y clientes.se
desea ser lo más familiar posible para el usuario,al saludar, despedirse y mostrar menús de opciones.\\
La interfaz y el menú de opciones  es requerido que cambie para cada rol,en donde cada respectivo menú contará con sus debidos servicios u opciones, y 
dado el servicio se le solicitará algunos requerimientos necesarios para llevarlo a cabo:
\begin{itemize}
    \item \textbf{Menú cliente(estandar)}
    \begin{itemize}
        \item Agendar cita medica 
        \begin{itemize}
            \item Nombre de la mascota
            \item Fecha y hora estimada para la cita
        \end{itemize}
        \item Agendar cita para Intervención quirúrgica
        \begin{itemize}
            \item Nombre de la mascota
            \item Fecha y hora estimada para la cita
            \item \textbf{Requerimiento:} Agendar cita para estudios clinicos (expres o tradicional)
        \end{itemize}
        No puede llevar a cabo una intervención quirúrgica si una mascota no se encuentra en óptimas codiciones.
        \item Agendar cita para estetica
        \begin{itemize}
            \item Nombre de la mascota
            \item Fecha y hora estimada para la cita
            \item Indicar tipo de corte de pelo.
        \end{itemize}
        \item Agendar cita para vacuna o desparacitación
        \begin{itemize}
            \item Nombre de la mascota
            \item Fecha y hora estimada para la cita
            \item Indicar nombre de la vacuna.
        \end{itemize}

        \item Agendar cita para estudios clinicos
            \begin{itemize}
                \item \textit{Estudios clinicos express}
                \begin{itemize}    
                    \item Nombre de la mascota
                    \item Fecha y hora estimada para la cita
                    \item Indicar problema a tratar.
                \end{itemize}
                Estos estudios se entregan al instante menos de 3 horas(3 segundos)
                \item \textit{Estudios clinicos tradicional}
                \begin{itemize}    
                    \item Nombre de la mascota
                    \item Fecha y hora estimada para la cita
                    \item Indicar problema a tratar.
                \end{itemize}
                Estos estudios se entregan dentro de 1 semana.
            \end{itemize}
     
        \item Pago de servicios
        \begin{itemize}
            \item Cita medica \$200
            \item Corte de pelo y uñas (estetica) \$300
            \item Vacunas \$250
            \item Intervención Quirúrgica \$1850
            \item Estudios clinicos
            \begin{itemize}
                  \item Estudios clinicos expres \$1000
                  \item Estudios clinicos tradicional \$800
                  \end{itemize}
        \end{itemize}
    \end{itemize}

    Al realizar el pago de cualquier servicio debe realizar la operación a través de un representante.El cliente deberá ingresar
    el número de cuenta bancaria,que se comparará con su debido atributo,en caso de no ser el mismo, el pago se cancelará y 
    regresará al menú estandar.\\
    Al completarse cualquier pago de servicio éxitoso el cliente podrá ver un ticket del respectivo servicio, la fecha agendada
    y al médico veterinario correspondiente.
\end{itemize}

\begin{itemize}
    \item \textbf{Menú Médico Veterinario}
    \begin{itemize}
        \item Citas proximas\\
               Se muestran las citas agendadas dada su categoria.  
        \item Clientes nuevos\\
               Se muentra el registro de los clientes nuevos.  
        \item Clientes con pagos éxitosos\\
              Se muestra aquellos clientes que ya pagaron el servicio solicitado
    \end{itemize}
\end{itemize}

 Al acceder al sitio se le mostrará un menú inicial en donde se le solicitará:

 \begin{itemize}
    \item \textbf{Menú inicial}
          \begin{itemize}
            \item ¿Primera vez que nos visitas?\\
                  En este primer apartado se debe registrar un cliente con atributos que debe de tener una mascota y un cliente.
            \item Soy cliente\\
                  En este apartado se da por hecho que el cliente ya se encuentra registrado, por lo que se
                  le solicitará sus debidas credenciales (usuario y contraseña) para acceder al sitio.  
            \item Soy médico veterinario\\
                  Al igual con el apartado anterior se da por hecho que el médico veterinario ya se encuentra 
                  registrado en el sitio por lo que tambien se le solicitará sus credenciales.
          \end{itemize}   
 \end{itemize}


%Bienvendo de nombre_mascota, que servicio estas buscando
\end{document}