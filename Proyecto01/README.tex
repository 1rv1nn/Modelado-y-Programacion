\documentclass[a4paper,12pt]{article} 
\usepackage[top=2cm,bottom=2cm,left=2cm,rigth=2cm,heightrounded]{geometry}
\usepackage[utf8]{inputenc}
\usepackage{multirow} 
\usepackage[spanish]{babel}
\usepackage[usenames]{color}
\usepackage{dsfont}
\usepackage{amssymb}
\usepackage{bbding}  
\usepackage[dvipsnames]{xcolor}
\usepackage{csquotes}
\usepackage[export]{adjustbox}
\usepackage[all]{nowidow} 
\usepackage{csquotes} 
\everymath{\displaystyle}
\usepackage{setspace}
\usepackage{hyperref}
\usepackage{multicol}
\usepackage[ddmmyyyy]{datetime} 
\renewcommand{\dateseparator}{-} 
\usepackage{fancyhdr}
\usepackage{amsmath,xcolor}
\usepackage[inline]{enumitem}
\usepackage{hhline,colortbl}
\usepackage[most]{tcolorbox}

%%%%%AQUÍ SE ENCUENTRAN LOS COLORES :)%%%%%%%%
\definecolor{babypink}{rgb}{0.96, 0.76, 0.76}
\definecolor{celadon}{rgb}{0.67, 0.88, 0.69}
\definecolor{outrageousorange}{rgb}{1.0, 0.43, 0.29}
\definecolor{spirodiscoball}{rgb}{0.06, 0.75, 0.99}
\definecolor{green(ncs)}{rgb}{0.0, 0.62, 0.42}
\definecolor{paleblue}{rgb}{0.69, 0.93, 0.93}
\definecolor{applegreen}{rgb}{0.55, 0.71, 0.0}
\definecolor{brightube}{rgb}{0.82, 0.62, 0.91}


\newtcolorbox{mybox}{colback=black!20!black,colframe=cyan!85!white}

\pagestyle{fancy} 
\fancyhead{}\renewcommand{\headrulewidth}{0pt} 
\fancyfoot[C]{} 
\fancyfoot[R]{\thepage} 
\newcommand{\note}[1]{\marginpar{\scriptsize \textcolor{red}{#1}}} 
\begin{document}
\fancyhead[C]{}
\begin{minipage}{0.295\textwidth} 
\raggedright
Equipo: Cerebros de Pollo\\    
\footnotesize 
\colorbox[rgb]{0.67, 0.88, 0.69}{Cruz González Irvin Javier}
\\\colorbox[rgb]{0.06, 0.75, 0.99 }{Ugalde Ubaldo Fernando}
\\\colorbox[rgb]{0.96, 0.76, 0.76}{Ugalde Flores Jimena}

\textcolor[rgb]{0.0, 0.72, 0.92}{\medskip\hrule}
\end{minipage}
\begin{minipage}{0.4\textwidth} 
\centering 
\large 
\textbf{Modelado y Programación}\\ 
\normalsize 
Tienda Virtual(CheemsMart) \\Proyecto 01\\
\end{minipage}
\begin{minipage}{0.295\textwidth} 
\raggedleft
\footnotesize
\colorbox[rgb]{0.67, 0.88, 0.69}{31716198-2}\\
\colorbox[rgb]{0.06, 0.75, 0.99 }{31714149-8}\\
\colorbox[rgb]{0.96, 0.76, 0.76}{31816536-1}\\
\today
\textcolor[rgb]{0.0, 0.72, 0.92}{\medskip\hrule}
\end{minipage}


\section*{Justificación}
Patrones de Diseño Utilizados:

   \begin{multicols}{4}
     \colorbox{pink}{\textit{Strategy}}

     \colorbox{brightube}{\textit{Facade}}

     \colorbox{applegreen}{\textit{Iterator}}

     \colorbox{cyan}{\textit{Proxy}}
   \end{multicols} 

    %En la parte de    -decidimos usar porque , -se resolvio de esta manera
   El programa recibe a un cliente con una identificación,por lo que deberá ingresar un usuario y contraseña especificos para poder ingresar 
   a la tienda.Dado el cliente,el idioma de la interfaz de inicio cambiará a ingles,español latino o español de españa.\\
   En esta última parte se optó por utilizar el patrón \colorbox{pink}{Strategy} ya que el comportamiento(idioma) cambia en tiempo de ejecución
   y permite que la interfaz cambie independiente del cliente que la use.De igual manera se hizo uso de \colorbox{pink}{Strategy} en la ofertas regionales de cada cliente dado que se cambia la estrategia(oferta) en cada respectivo 
   cliente.\\

   Para presentar el catálogo de forma remota del catálogo real se hizo uso del patrón \colorbox{brightube}{Facade} donde se desacopla al cliente del catálogo real de igual forma para que el sistema muestre ofertas
   a todos los usuarios adecuados. \\

   Para el manejo de los productos del catálogo se hizo uso del patrón \colorbox{applegreen}{Iterator} dado que proporciona una forma de acceder a los elementos secuencialmente de la estructura arrayList sin exponer su representación
   básica NO se hizo uso de \textit{Composite} porqué no necesitamos tener una jerarquía de los productos ni que se traten de manera uniforme o individuales.\\

   En el apartado de compra segura,donde se realiza cualquier operación a través de un representante se uso el patrón \colorbox{cyan}{Proxy} debido a que este representante o sustituto permite controlar el 
   acceso a los datos personales de un cliente y tiene la necesidad de dar una referencia al número de cuenta. 


\end{document}   
