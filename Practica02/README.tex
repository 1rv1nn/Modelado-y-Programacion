\documentclass[a4paper,10pt]{article} 
\usepackage[top=2cm,bottom=2cm,left=2cm,rigth=2cm,heightrounded]{geometry}
\usepackage[utf8]{inputenc}
\usepackage{multirow} 
\usepackage[spanish]{babel}
\usepackage[usenames]{color}
\usepackage{dsfont}
\usepackage{amssymb}
\usepackage{bbding}  
\usepackage[dvipsnames]{xcolor}
\usepackage{csquotes}
\usepackage[export]{adjustbox}
\usepackage[all]{nowidow} 
\usepackage{csquotes} 
\everymath{\displaystyle}
\usepackage{setspace}
\usepackage[ddmmyyyy]{datetime} 
\renewcommand{\dateseparator}{-} 
\usepackage{fancyhdr}
\usepackage{amsmath,xcolor}
\usepackage[inline]{enumitem}
\usepackage{hhline,colortbl}
\usepackage[most]{tcolorbox}

%%%%%AQUÍ SE ENCUENTRAN LOS COLORES :)%%%%%%%%
\definecolor{babypink}{rgb}{0.96, 0.76, 0.76}
\definecolor{celadon}{rgb}{0.67, 0.88, 0.69}
\definecolor{outrageousorange}{rgb}{1.0, 0.43, 0.29}
\definecolor{spirodiscoball}{rgb}{0.06, 0.75, 0.99}
\definecolor{green(ncs)}{rgb}{0.0, 0.62, 0.42}
\definecolor{paleblue}{rgb}{0.69, 0.93, 0.93}


\newtcolorbox{mybox}{colback=black!20!black,colframe=cyan!85!white}

\pagestyle{fancy} 
\fancyhead{}\renewcommand{\headrulewidth}{0pt} 
\fancyfoot[C]{} 
\fancyfoot[R]{\thepage} 
\newcommand{\note}[1]{\marginpar{\scriptsize \textcolor{red}{#1}}} 
\begin{document}
\fancyhead[C]{}
\begin{minipage}{0.295\textwidth} 
\raggedright
Equipo: Cerebros de Pollo\\    
\footnotesize 
\colorbox[rgb]{0.67, 0.88, 0.69}{Cruz González Irvin Javier}
\\\colorbox[rgb]{0.06, 0.75, 0.99 }{Ugalde Ubaldo Fernando}
\\\colorbox[rgb]{0.96, 0.76, 0.76}{Ugalde Flores Jimena}

\textcolor[rgb]{0.0, 0.72, 0.92}{\medskip\hrule}
\end{minipage}
\begin{minipage}{0.4\textwidth} 
\centering 
\large 
\textbf{Modelado y Programación}\\ 
\normalsize 
State, Template e Iterator \\Practica 02\\
\end{minipage}
\begin{minipage}{0.295\textwidth} 
\raggedleft
\footnotesize
\colorbox[rgb]{0.67, 0.88, 0.69}{31716198-2}\\
\colorbox[rgb]{0.06, 0.75, 0.99 }{31714149-8}\\
\colorbox[rgb]{0.96, 0.76, 0.76}{31816536-1}\\
\today
\textcolor[rgb]{0.0, 0.72, 0.92}{\medskip\hrule}
\end{minipage}

\begin{enumerate}
    
    \item Menciona los principios de diseño esenciales del patrón State, Template e Iterator. Menciona una desventaja de cada patrón.\\\\ De acuerdo a los principios SOLID
        \begin{enumerate}
            \item \colorbox[rgb]{0.53, 0.81, 0.98}{State}
                \begin{itemize}
                    \item Sustitución de Liskov
                    \item Principio de segregación de interfaces.\\Dado que este principio se basa principalmente en dividir varias interfaces
                          donde definen comportamientos más especificos es decir no obligar a ninguna clase a implementar métodos que no utiliza;
                          State aplica de una manera eficiente este comportamiento dado que se tiene una interfaz con cada uno de los estados del 
                          objecto de contexto.
                    \item State puede ser usado cuando necesitamos cambiar el estado de un objeto en tiempo de ejecución,teniendo diferentes subclases de algunas  
                    clases; Esta circunstancia es una ventaja y \textcolor{red}{desventaja} al mismo tiempo, porque tenemos clases con estados claros con cierta lógica y, por otro lado, el número de clases crece.
                   
                \end{itemize}


            
            \item \colorbox[rgb]{1.0, 0.71, 0.76}{Template}
            \begin{itemize}
                \item 
                \item 
                 \\
                
            \end{itemize}
            

            \item \colorbox[rgb]{0.69, 0.61, 0.85}{Iterator}
            \begin{itemize}
                \item %Sustitución de Liskov
                \item        
            \end{itemize}    
        \end{enumerate}
\end{enumerate}

\begin{itemize}
    \item \textbf{OBSERVACIONES IMPORTANTES}
\end{itemize}
\end{document}   
