\documentclass[a4paper,10pt]{article} 
\usepackage[top=2cm,bottom=2cm,left=2cm,rigth=2cm,heightrounded]{geometry}
\usepackage[utf8]{inputenc}
\usepackage{multirow} 
\usepackage[spanish]{babel}
\usepackage[usenames]{color}
\usepackage{dsfont}
\usepackage{amssymb}
\usepackage{bbding}  
\usepackage[dvipsnames]{xcolor}
\usepackage{csquotes}
\usepackage[export]{adjustbox}
\usepackage[all]{nowidow} 
\usepackage{csquotes} 
\everymath{\displaystyle}
\usepackage{setspace}
\usepackage[ddmmyyyy]{datetime} 
\renewcommand{\dateseparator}{-} 
\usepackage{fancyhdr}
\usepackage{amsmath,xcolor}
\usepackage[inline]{enumitem}
\usepackage{hhline,colortbl}
\usepackage[most]{tcolorbox}

%%%%%AQUÍ SE ENCUENTRAN LOS COLORES :)%%%%%%%%
\definecolor{babypink}{rgb}{0.96, 0.76, 0.76}
\definecolor{celadon}{rgb}{0.67, 0.88, 0.69}
\definecolor{outrageousorange}{rgb}{1.0, 0.43, 0.29}
\definecolor{spirodiscoball}{rgb}{0.06, 0.75, 0.99}
\definecolor{green(ncs)}{rgb}{0.0, 0.62, 0.42}
\definecolor{paleblue}{rgb}{0.69, 0.93, 0.93}


\newtcolorbox{mybox}{colback=black!20!black,colframe=cyan!85!white}

\pagestyle{fancy} 
\fancyhead{}\renewcommand{\headrulewidth}{0pt} 
\fancyfoot[C]{} 
\fancyfoot[R]{\thepage} 
\newcommand{\note}[1]{\marginpar{\scriptsize \textcolor{red}{#1}}} 
\begin{document}
\fancyhead[C]{}
\begin{minipage}{0.295\textwidth} 
\raggedright
Equipo: Cerebros de Pollo\\    
\footnotesize 
\colorbox[rgb]{0.67, 0.88, 0.69}{Cruz González Irvin Javier}
\\\colorbox[rgb]{0.06, 0.75, 0.99 }{Ugalde Ubaldo Fernando}
\\\colorbox[rgb]{0.96, 0.76, 0.76}{Ugalde Flores Jimena}

\textcolor[rgb]{0.0, 0.72, 0.92}{\medskip\hrule}
\end{minipage}
\begin{minipage}{0.4\textwidth} 
\centering 
\large 
\textbf{Modelado y Programación}\\ 
\normalsize 
Decorator y Adapter \\Practica 03\\
\end{minipage}
\begin{minipage}{0.295\textwidth} 
\raggedleft
\footnotesize
\colorbox[rgb]{0.67, 0.88, 0.69}{31716198-2}\\
\colorbox[rgb]{0.06, 0.75, 0.99 }{31714149-8}\\
\colorbox[rgb]{0.96, 0.76, 0.76}{31816536-1}\\
\today
\textcolor[rgb]{0.0, 0.72, 0.92}{\medskip\hrule}
\end{minipage}

\begin{enumerate}
    
    \item Menciona los principios de diseño esenciales de los patrones Decorator y Adapter. Menciona una desventaja de cada patrón.\\\\
    
        Dado que ambos patrones pertencen a la misma categoría (Patrones Estructurales) se cumplen los siguientes principios:
        \begin{enumerate}
            \item \colorbox[rgb]{0.53, 0.81, 0.98}{Decorator}
            \begin{itemize}
               \item \textit{Principio de responsabilidad única}
               
                      Permite dividir funcionalidad entre clases con áreas de interés únicas. 
               
               \item \textit{Principio de abierto cerrado}  

                \item Proporciona una manera más flexible de añadir responsabilidades a los objetos que la que podia obtenerse a través de la herencia(multiple) estática, teniendo como una \textit{desventaja} que se 
                requiere crear una nueva clase para cada responsabilidad adicional, esto da lugar a muchas clases diferentes e incrementa la complejidad de un sistema.Heredar dos veces una clase resulta cuanto menos, propenso a errores             
                      
            \end{itemize}

            
            \item \colorbox[rgb]{1.0, 0.71, 0.76}{Adapter}
            \begin{itemize}
                \item \textit{Principio de responsabilidad única}
                
                      Puede separar la interfaz o el código de conversión de datos de la lógica de negocio primaria del programa\\\\  

                \item \textit{Principio de abierto cerrado}
                
                      Introduce nuevos tipos de adaptadores al programa sin descomponer el código cliente existente, siempre y cuando trabajen con los
                      adaptadores a tráves de la interfaz con el cliente.\\\\  
                     
                \item  Una \textit{desventaja} es que la complejidad general del código aumenta, ya que se deben introducir un grupo de nuevas interfaces y clases.
                      En ocasión resulta más sencillo cambiar la clase de servicios de modo que coincida con el resto del código.         
            \end{itemize}              
        \end{enumerate}

        \begin{thebibliography}{0}
            \bibitem{Mauricio2016} Gamma, E., Helm, R., Johnson, R. & Vlissides, J. (1994). Design Patterns:Elements of Reusable Object-Oriented Software.                           
          \end{thebibliography}
        
       
        
                 

\end{enumerate}
\end{document}   
